\documentclass[11pt]{article}
\usepackage{cs65f12}
\usepackage{times}
\usepackage{latexsym}
\usepackage{ulem}
\setlength\titlebox{6.5cm}    % Expanding the titlebox

\title{BatTrace: Android battery performance testing via system call tracing}

\author{Yeayeun Park\\
  {\tt ypark2@swarthmore.edu}  
  \And                           
  Mark Serrano\\
  {\tt mserran2@swarthmore.edu}
  \And                            
  Craig Pentrack\\                 
  {\tt cpentra1@swarthmore.edu}}

\date{10/04/13}

\begin{document}
\maketitle
\begin{abstract}
  The increasing number of tasks we can perform on our mobile devices 
  feeds positively into the demand for devices with longer battery power. 
  While the rate of increase in the usefulness of the devices is 
  growing substantially faster than the rate of improvement in the battery 
  performance that is achieved in the market, it has become a rising issue 
  for mobile users how to use their devices efficiently and wisely, given 
  the limitations. In this paper, we examine Android devices in particular, 
  and present BatTrace, an Android analysis platform that evaluates Android 
  battery performance by tracing system calls. BatTrace will execute 
  different types of popular and important system calls, and extract the 
  correlation between a particular system call and its influence on the 
  battery. Subsequently, it will trace system calls made by individual Android 
  applications, and the resulting data will be used to profile each application. 
  Finally, the analysis on the correlation between system calls and their 
  battery usage, and that on the correlation between each application and system 
  calls they initiate will be combined to produce the estimated battery performance 
  of the Android applications.
\end{abstract}

\section{Motivation}

Our project is motivated by an issue that we face daily: limited battery power 
of our mobile devices. The variety of tasks we can perform on our mobile
devices is countered by the amount of battery that we're allowed to use. Given 
that dynamic analysis executes data in real-time to evaluate and test programs, 
we searched for tools that we could potentially perform dynamic analysis on our 
mobile devices with, and uncover the low-level explanation for what really is 
draining the battery. By profiling particular system calls in temrs of their battery
usage, we're hoping to dervice a correlation between the two. Subsequently, with tools 
such as 'strace' that suits our purpose, we plan to trace system calls made by third 
party applications, and, in turn, provide a good set of guidelines for mobile users 
to follow, when the low battery crisis comes.

\section{Background}

Historically much power consumption research has focused on using utilization-based
methods.  However, modern smartphones employ complex power strategies in device 
drivers and OS-level power management, sometimes rendering utilization as a poor 
model for representing power states and deducing battery usage [1].  While 
sometimes strong correlation exists between utilization and power consumption, often 
applications have constant power consumption while in certain states (while 
utilization fluctuates) or have high power consumption while low utilization [1], [2].  
Additionally, measuring utilization via performance counters results in accuracy 
loss [1].  Instead of modeling power with utilization, system calls, the only way of 
interacting with hardware and performing I/O, serve as a much more precise indicator 
of power consumption [1].  Past work and tools, such as eProf, have shown systems 
calls to be an effective way of modeling power [1], [3], [4], [6].  Using the findings 
of eProf and other studies as justification, we plan to measure and classify system 
calls on Android smartphones in terms of their effect on battery life.

While eProf foregrounded system calls as an effective indicator of changes in power
state, eProf used systems calls as means to profiling applications with regard to power 
consumption on a sub-routine level [1], [4].  Developing models based off of system 
calls supplied a powerful tool, however the eProf research did not study battery drain 
as a result of  particular system calls themselves and the frequency with which 
applications rely on certain system calls.  Other work in the smartphone battery life 
research, including detecting energy-related bugs, correlating wireless signal strength 
with battery consumption, and generating battery usage information on the process or 
application level, has relied on system calls. [3] [5] [6].  We plan to supplement the 
research area by focusing our study on the system calls themselves, rather than using 
them as a means in tracking changes in power state, detecting bugs, measuring signal 
strength effects, or producing higher-level profilings as explored previously.


\section{Our Idea}

While dynamic analysis on traditional devices involves the most efficient use of finite 
computing resources, mobile devices introduce a new problem; finite power. The issue we 
immediately encounter when trying to analyze mobile software applications is that we 
\sout{almost} never have access to the source code of the applications. This is 
especially true given the fact that most mobile software is proprietary in nature, 
leaving open source software to the relics that are desktop computers.

With this in mind we set out find a way of measuring mobile battery usage at very low level
(software wise). We decided a good approach would involve monitoring activity at the system 
call level using a tool like strace. Ideally, we want to profile a variety of system calls 
based on how much battery is used while they are running. We intend to establish a baseline 
battery consumption level so we know how much battery is used by just the OS. Then, using 
simple programs that repeatedly make the same system call X times, we can determine how much 
battery was used as a result of initiating a particular system call X times.

Once system calls have been profiled, we can proceed to the last phase of the analysis. Our 
goal is to identify the system calls initiated by the Dalvik VM as a result of running an 
individual app. By identifying the types of system calls, as well as the number of calls made 
to an individual system call, we will be able to predict the app's impact on the battery based 
on what we learned about battery usage for individual system calls. While this approach may not 
be the most accurate, we believe it is the broadest approach that will allow us to profile any 
application regardless of the author or the nature of the software's license.


Manuscripts must be in two-column format.  Exceptions to the
two-column format include the title, authors' names and complete
addresses, which must be centered at the top of the first page, and
any full-width figures or tables (see the guidelines in
Subsection~\ref{ssec:first}). {\bf Type single-spaced.}  Start all
pages directly under the top margin. See the guidelines later
regarding formatting the first page.

\subsection{Electronically-available resources}

This description is provided in \LaTeX{} (\nobreak{cs65f12.tex}) along
with the \LaTeX{} style file used to format it
(\nobreak{cs65f12.sty}).  In addition, there is a bibliography style
(\nobreak{cs65f12.bst}) and sample bibliography file
(\nobreak{cs65f12.bib}).  These files are all in the {\tt cs65/labs/04-05/}
directory.

\subsection{Format of Electronic Manuscript}
\label{sect:pdf}

The easiest way to turn this \LaTeX{} into a PDF, is to use the
Makefile found (also in {\tt cs65/labs/04-05/}).  The Makefile will
compile your file (and your bibliography file) and turn it into a PDF.

Here are the basic instructions:
\begin{itemize}
\item {\tt make} will create a PDF file from your \LaTeX{} document.
\item {\tt make view} will display the PDF file. 
\item {\tt make clean} will clean up some files you might not need
\item {\tt make cleanall} will clean up all non-source files
\end{itemize}

For reasons of uniformity, Adobe's {\bf Times Roman} font should be
used. In \LaTeX{} this is accomplished by putting

\begin{quote}
\begin{verbatim}
\usepackage{times}
\usepackage{latexsym}
\end{verbatim}
\end{quote}
in the preamble, as was done in this file.

Print-outs of the PDF file should look like the present document,
which conforms to the formatting requirements. If you cannot meet the
above requirements, please contact me as soon as possible.

\subsection{Layout}
\label{ssec:layout}

Format manuscripts two columns to a page, in the manner these
instructions are formatted. The exact dimensions for a page on US-letter
paper are:

\begin{itemize}
\item Left and right margins: 1in
\item Top margin:1in
\item Bottom margin: 1in
\item Column width: 3.15in
\item Column height: 9in
\item Gap between columns: 0.2in
\end{itemize}

Papers should not be submitted on any other paper size. 

\subsection{The First Page}
\label{ssec:first}

Center the title, author's name(s) and affiliation(s) across both
columns. Do not use footnotes for affiliations.  Use the two-column
format only when you begin the abstract.

{\bf Title}: Place the title centered at the top of the first page, in
a 15-point bold font. A long title should be typed on two lines
without a blank line intervening. Approximately, put the title at 1in
from the top of the page, followed by a blank line, then the author's
names(s), and the affiliation on the following line.  Do not use only
initials for given names (middle initials are allowed). The
affiliation should contain the author's complete address, and an email
address. Leave about 0.75in between the affiliation and the body of
the first page.

{\bf Abstract}: Type the abstract at the beginning of the first
column.  The width of the abstract text should be smaller than the
width of the columns for the text in the body of the paper by about
0.25in on each side.  Center the word {\bf Abstract} in a 12 point
bold font above the body of the abstract. The abstract should be a
concise summary of the general thesis and conclusions of the paper.
It should be no longer than 200 words.

{\bf Text}: Begin typing the main body of the text immediately after
the abstract, observing the two-column format as shown in 
the present document.

{\bf Indent} when starting a new paragraph. For reasons of uniformity,
use Adobe's {\bf Times Roman} fonts, with 11 points for text and 
subsection headings, 12 points for section headings and 15 points for
the title. If Times Roman is unavailable, use {\bf Computer Modern
  Roman} (\LaTeX{}'s default; see section \ref{sect:pdf} above).
Note that the latter is about 10\% less dense than Adobe's Times Roman
font.

\subsection{Sections}

{\bf Headings}: Type and label section and subsection headings in the
style shown on the present document.  Use numbered sections (Arabic
numerals) in order to facilitate cross references. Number subsections
with the section number and the subsection number separated by a dot,
in Arabic numerals. Do not number subsubsections.

{\bf Citations}: Citations within the text appear in parentheses
as~\cite{harris1955-phoneme} or, if the author's name appears in the
text itself, as Harris~\shortcite{harris1955-phoneme}.  Append
lowercase letters to the year in cases of ambiguities.  Treat double
authors as in~\cite{hafer1974-word}, but write as
in~\cite{hana2006-tagging} when more than two authors are involved.
Collapse multiple citations as
in~\cite{harris1967-morpheme,dejean1998-morphemes}.

\textbf{References}: Gather the full set of references together under
the heading {\bf References}; place the section before any Appendices,
unless they contain references. Arrange the references alphabetically
by first author, rather than by order of occurrence in the text.
Provide as complete a citation as possible, using a consistent format.

The \LaTeX{} and Bib\TeX{} style files provided roughly fit the
American Psychological Association format, allowing regular citations, 
short citations and multiple citations as described above.

{\bf Appendices}: Appendices, if any, directly follow the text and the
references (but see above).  Letter them in sequence and provide an
informative title: {\bf Appendix A. Title of Appendix}.

\textbf{Acknowledgement} sections should go as a last section immediately
before the references.  Do not number the acknowledgement section.

\subsection{Footnotes}

{\bf Footnotes}: Put footnotes at the bottom of the page. They may
be numbered or referred to by asterisks or other
symbols.\footnote{This is how a footnote should appear.} Footnotes
should be separated from the text by a line.\footnote{Note the
line separating the footnotes from the text.}

\subsection{Graphics}

{\bf Illustrations}: Place figures, tables, and photographs in the
paper near where they are first discussed, rather than at the end, if
possible.  Wide illustrations may run across both columns. Do not use
color illustrations as they may reproduce poorly.

{\bf Captions}: Provide a caption for every illustration; number each one
sequentially in the form:  ``Figure 1. Caption of the Figure.'' ``Table 1.
Caption of the Table.''  Type the captions of the figures and 
tables below the body, using 11 point text.  


\bibliographystyle{cs65f12}
\bibliography{cs65f12}

\end{document}
